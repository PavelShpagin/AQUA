\documentclass[11pt]{article}
\usepackage[utf8]{inputenc}
\usepackage[margin=1in]{geometry}
\usepackage{amsmath,amssymb}
\usepackage{booktabs}
\usepackage{graphicx}
\usepackage{hyperref}
\usepackage{enumitem}
\usepackage{xcolor}
\usepackage{tikz}
\usepackage{natbib}

\title{\textbf{AQUA Paper Publication Plan} \\
\large Strategic Roadmap for ACL 2025 and Top NLP Venues}
\author{Paper Planning Document}
\date{\today}

\begin{document}

\maketitle

\tableofcontents
\newpage

\section{Target Venues and Timeline}

\subsection{Primary Target: ACL 2025 Main Conference}

\begin{itemize}
    \item \textbf{Venue}: ACL 2025 (Annual Meeting of the Association for Computational Linguistics)
    \item \textbf{Expected Deadline}: February 2025 (ARR submission) or direct submission
    \item \textbf{Notification}: April-May 2025
    \item \textbf{Conference}: July 2025 (Vienna, Austria)
    \item \textbf{Track}: Main Conference (Long Paper, 8 pages + references)
\end{itemize}

\subsection{Alternative Venues (in order of preference)}

\begin{enumerate}
    \item \textbf{EMNLP 2025}: Empirical Methods in NLP (Deadline: May 2025)
    \item \textbf{NAACL 2025}: North American Chapter of ACL (Deadline: October 2024 for NAACL 2025)
    \item \textbf{COLING 2025}: International Conference on Computational Linguistics
    \item \textbf{EACL 2025}: European Chapter of ACL
    \item \textbf{*SEM 2025}: Joint Conference on Lexical and Computational Semantics
\end{enumerate}

\subsection{Workshop Fallback Options}

\begin{itemize}
    \item \textbf{BEA Workshop}: Building Educational Applications (co-located with ACL/EMNLP)
    \item \textbf{INLG}: International Natural Language Generation Conference
    \item \textbf{MultiLing Workshop}: Multilingual Text Processing
\end{itemize}

\section{Proposed Paper Titles}

\subsection{Primary Title Options}

\begin{enumerate}
    \item \textbf{AQUA: Autonomous Quality Assurance for Multilingual Grammatical Error Correction}
    
    \item \textbf{Dynamic Ensemble Methods for LLM-as-a-Judge in Grammatical Error Correction}
    
    \item \textbf{Beyond Binary: Fine-Grained GEC Evaluation with Agentic LLM Judges}
    
    \item \textbf{Agent-as-a-Judge: Tool-Augmented Reasoning for GEC Quality Assessment}
\end{enumerate}

\subsection{Recommended Title}

\begin{center}
\fbox{\parbox{0.9\textwidth}{
\centering
\textbf{AQUA: Autonomous Quality Assurance for Multilingual GEC via Agentic Ensemble Judging}
}}
\end{center}

\section{Paper Structure and Outline}

\subsection{Abstract (250 words)}

\begin{quote}
\textit{Outline}: Introduce the GEC evaluation challenge $\rightarrow$ Present AQUA framework $\rightarrow$ Highlight key innovations (6-class taxonomy, dynamic escalation, inner debate, agent-as-judge) $\rightarrow$ Report main results (93\% 4-class accuracy, +3\% from agentic methods) $\rightarrow$ Emphasize multilingual and production-ready aspects.
\end{quote}

\subsection{Section 1: Introduction (1 page)}

\begin{enumerate}
    \item \textbf{Problem Motivation}
    \begin{itemize}
        \item GEC systems produce suggestions ranging from beneficial to harmful
        \item Manual evaluation is expensive and doesn't scale
        \item Existing automated metrics (ERRANT, GLEU) don't capture severity
        \item Need for fine-grained, explainable quality assessment
    \end{itemize}
    
    \item \textbf{Contributions Preview}
    \begin{itemize}
        \item Novel 6-class taxonomy (TP, FP1, FP2, FP3, TN, FN)
        \item Dynamic escalation ensemble for cost-efficient routing
        \item Inner Debate ensemble for structured disagreement resolution
        \item Agent-as-a-Judge with tool-augmented reasoning
        \item Multilingual support and production-ready implementation
    \end{itemize}
    
    \item \textbf{Paper Organization}
\end{enumerate}

\subsection{Section 2: Related Work (1 page)}

\begin{enumerate}
    \item \textbf{GEC Evaluation Methods}
    \begin{itemize}
        \item Reference-based: M$^2$, ERRANT, GLEU
        \item Reference-free: Grammaticality models, LLM-based scoring
        \item Human evaluation protocols
    \end{itemize}
    
    \item \textbf{LLM-as-a-Judge}
    \begin{itemize}
        \item G-Eval, GPTScore for NLG evaluation
        \item MT-Bench, Arena for instruction following
        \item Self-consistency and calibration methods
    \end{itemize}
    
    \item \textbf{Ensemble Methods for NLP}
    \begin{itemize}
        \item Voting and weighted aggregation
        \item Mixture of Experts (MoE)
        \item Debate and argumentation approaches
    \end{itemize}
    
    \item \textbf{Agentic NLP Systems}
    \begin{itemize}
        \item ReAct, Tool-augmented LLMs
        \item RAG for knowledge grounding
        \item Multi-agent systems
    \end{itemize}
\end{enumerate}

\subsection{Section 3: Task Definition and Taxonomy (0.75 pages)}

\begin{enumerate}
    \item \textbf{Problem Formulation}
    \begin{itemize}
        \item Input: (source sentence, GEC suggestion)
        \item Output: Quality label with confidence and reasoning
        \item Objective: Maximize accuracy while minimizing cost
    \end{itemize}
    
    \item \textbf{Label Taxonomy}
    \begin{itemize}
        \item TP: True Positive (beneficial correction)
        \item FP1: Critical False Positive (meaning change, sensitivity)
        \item FP2: Medium False Positive (introduces error)
        \item FP3: Minor False Positive (stylistic, optional)
        \item TN: True Negative (correctly unchanged)
        \item FN: False Negative (missed correction)
    \end{itemize}
    
    \item \textbf{Evaluation Protocols}
    \begin{itemize}
        \item 6-class multiclass evaluation
        \item 4-class (TP/FP1/FP2/FP3) for edit-level
        \item Binary TP/FP aggregation
        \item Simplified TP/FP (FP3$\rightarrow$TP)
    \end{itemize}
\end{enumerate}

\subsection{Section 4: AQUA Framework (2.5 pages)}

\begin{enumerate}
    \item \textbf{4.1 System Overview}
    \begin{itemize}
        \item Architecture diagram
        \item Modular judge types (Feedback, Sentence, Edit, TN/FN)
        \item ERRANT alignment integration
    \end{itemize}
    
    \item \textbf{4.2 Baseline Judge}
    \begin{itemize}
        \item Prompt design with explicit decision order
        \item Trigger tests for each category
        \item JSON output parsing
    \end{itemize}
    
    \item \textbf{4.3 Dynamic Escalation Ensemble}
    \begin{itemize}
        \item Risk flag computation (language-agnostic heuristics)
        \item Routing logic: small $\rightarrow$ mid $\rightarrow$ top
        \item Finalizer arbitration for conflicts
        \item Cost-efficiency analysis
    \end{itemize}
    
    \item \textbf{4.4 Iterative Critic Ensemble}
    \begin{itemize}
        \item Dual buffer mechanism (TP/FP vs TN/FN)
        \item Consensus checking algorithm
        \item Opinion formatting and context passing
        \item Top-2 class selection for final judgment
    \end{itemize}
    
    \item \textbf{4.5 Inner Debate Ensemble}
    \begin{itemize}
        \item Dominance priority hierarchy
        \item Debater selection (balanced min(m,k))
        \item Alternating argument structure
        \item Debate-aware final judgment prompts
    \end{itemize}
    
    \item \textbf{4.6 Agent-as-a-Judge}
    \begin{itemize}
        \item Tool inventory (Grammar RAG, Meaning Change, Nonsense, Quality)
        \item ReAct reasoning framework
        \item Language-specific grammar databases
        \item Tool invocation and result synthesis
    \end{itemize}
\end{enumerate}

\subsection{Section 5: Experimental Setup (0.75 pages)}

\begin{enumerate}
    \item \textbf{Datasets}
    \begin{itemize}
        \item Gold datasets: English, German, Ukrainian, Spanish
        \item Data collection and annotation protocol
        \item Inter-annotator agreement
        \item Train/dev/test splits
    \end{itemize}
    
    \item \textbf{Baselines}
    \begin{itemize}
        \item Single-model baselines (GPT-4, GPT-4o-mini, etc.)
        \item Simple ensemble (majority voting)
        \item Prior LLM-as-Judge methods (G-Eval adaptation)
    \end{itemize}
    
    \item \textbf{Models and Backends}
    \begin{itemize}
        \item Small: gpt-4.1-nano, gemini-2.0-flash-lite
        \item Medium: gpt-4o-mini, claude-3-haiku
        \item Large: gpt-4o, o3, claude-3-opus
    \end{itemize}
    
    \item \textbf{Evaluation Metrics}
    \begin{itemize}
        \item Accuracy, Precision, Recall, F1 (macro and weighted)
        \item Per-class metrics
        \item Cost per 1000 evaluations
        \item Latency analysis
    \end{itemize}
\end{enumerate}

\subsection{Section 6: Results (1.5 pages)}

\begin{enumerate}
    \item \textbf{6.1 Main Results}
    \begin{itemize}
        \item Table: 6-class accuracy across methods and languages
        \item Table: Binary TP/FP results
        \item Analysis: Which methods excel on which classes
    \end{itemize}
    
    \item \textbf{6.2 Escalation Efficiency}
    \begin{itemize}
        \item Escalation rate analysis
        \item Cost vs. accuracy tradeoff curves
        \item Per-category escalation patterns
    \end{itemize}
    
    \item \textbf{6.3 Agent Performance}
    \begin{itemize}
        \item Tool usage statistics
        \item RAG query effectiveness
        \item Comparison with non-agentic approaches
    \end{itemize}
    
    \item \textbf{6.4 Ablation Studies}
    \begin{itemize}
        \item Effect of number of judges
        \item Impact of different backends
        \item Contribution of each ensemble component
    \end{itemize}
    
    \item \textbf{6.5 Cross-lingual Analysis}
    \begin{itemize}
        \item Performance variation across languages
        \item Language-specific error patterns
        \item Transfer learning potential
    \end{itemize}
\end{enumerate}

\subsection{Section 7: Analysis and Discussion (0.75 pages)}

\begin{enumerate}
    \item \textbf{Error Analysis}
    \begin{itemize}
        \item Common failure modes
        \item FP1 vs FP3 confusion cases
        \item Edge cases and ambiguities
    \end{itemize}
    
    \item \textbf{Qualitative Examples}
    \begin{itemize}
        \item Success cases: Inner Debate resolving disagreement
        \item Failure cases: Agent tool misuse
        \item Escalation trigger effectiveness
    \end{itemize}
    
    \item \textbf{Limitations}
    \begin{itemize}
        \item Dependence on LLM capabilities
        \item Cost constraints for high-volume deployment
        \item Language coverage gaps
    \end{itemize}
\end{enumerate}

\subsection{Section 8: Conclusion (0.25 pages)}

\begin{enumerate}
    \item Summary of contributions
    \item Key takeaways for practitioners
    \item Future work directions
\end{enumerate}

\section{Key Figures and Tables}

\subsection{Required Figures}

\begin{enumerate}
    \item \textbf{Figure 1}: AQUA System Architecture (full-page diagram)
    \item \textbf{Figure 2}: Dynamic Escalation Flow
    \item \textbf{Figure 3}: Inner Debate Example
    \item \textbf{Figure 4}: Agent ReAct Reasoning Trace
    \item \textbf{Figure 5}: Cost vs. Accuracy Tradeoff Curves
\end{enumerate}

\subsection{Required Tables}

\begin{enumerate}
    \item \textbf{Table 1}: Dataset Statistics
    \item \textbf{Table 2}: Main Results (6-class accuracy)
    \item \textbf{Table 3}: Binary TP/FP Results
    \item \textbf{Table 4}: Per-class Precision/Recall/F1
    \item \textbf{Table 5}: Escalation Rate and Cost Analysis
    \item \textbf{Table 6}: Ablation Study Results
    \item \textbf{Table 7}: Cross-lingual Performance
\end{enumerate}

\section{Novelty Claims}

\subsection{Primary Novelty}

\begin{enumerate}
    \item \textbf{First agentic LLM-as-Judge for GEC}: No prior work combines tool-augmented reasoning with ensemble methods for GEC evaluation.
    
    \item \textbf{Inner Debate Ensemble}: Novel debate-style resolution mechanism that structures disagreements into formal arguments before adjudication.
    
    \item \textbf{Dynamic Escalation with Deterministic Routing}: Unique combination of heuristic risk flags with LLM-based classification for cost-efficient routing.
    
    \item \textbf{Fine-grained 6-class Taxonomy}: Extension beyond binary TP/FP to capture severity gradations critical for production deployment.
\end{enumerate}

\subsection{Supporting Claims}

\begin{enumerate}
    \item Dual-buffer consensus mechanism in Iterative Critic
    \item Language-agnostic risk flag computation
    \item Multilingual grammar RAG databases
    \item Production-ready implementation with cost tracking
\end{enumerate}

\section{Comparison to Related Work}

\subsection{vs. LLM-as-a-Judge Methods}

\begin{itemize}
    \item \textbf{G-Eval} (Liu et al., 2023): Single-model, no ensemble, no tool use
    \item \textbf{GPTScore} (Fu et al., 2023): Probability-based, no fine-grained taxonomy
    \item \textbf{AQUA}: Ensemble + agentic + 6-class + multilingual
\end{itemize}

\subsection{vs. GEC Evaluation Methods}

\begin{itemize}
    \item \textbf{ERRANT} (Bryant et al., 2017): Rule-based, no severity assessment
    \item \textbf{GLEU} (Napoles et al., 2015): Reference-based, binary correct/incorrect
    \item \textbf{AQUA}: Reference-free, severity-aware, explainable
\end{itemize}

\subsection{vs. Multi-Agent Systems}

\begin{itemize}
    \item \textbf{AutoGen} (Wu et al., 2023): General conversation, not GEC-specific
    \item \textbf{Debate} (Irving et al., 2018): Theoretical framework, not implemented for NLP tasks
    \item \textbf{AQUA}: GEC-specialized with grammar tools and debate-style resolution
\end{itemize}

\section{Experimental Gaps to Fill}

\subsection{Critical Experiments Needed}

\begin{enumerate}
    \item \textbf{Inter-annotator Agreement Study}
    \begin{itemize}
        \item Compute Cohen's Kappa or Krippendorff's Alpha
        \item Per-class agreement analysis
        \item Annotator calibration protocol
    \end{itemize}
    
    \item \textbf{Statistical Significance Testing}
    \begin{itemize}
        \item Bootstrap confidence intervals
        \item Paired t-tests for method comparisons
        \item Multiple comparison correction
    \end{itemize}
    
    \item \textbf{Human Evaluation}
    \begin{itemize}
        \item Sample of AQUA predictions rated by humans
        \item Correlation with human judgments
        \item Reasoning quality assessment
    \end{itemize}
    
    \item \textbf{Additional Languages}
    \begin{itemize}
        \item French, Italian (Romance languages)
        \item Chinese, Japanese (non-Indo-European)
        \item Low-resource language testing
    \end{itemize}
\end{enumerate}

\subsection{Ablation Experiments}

\begin{enumerate}
    \item Number of judges: 1, 2, 3, 5, 7
    \item Escalation thresholds: vary risk flag sensitivity
    \item Debate rounds: 1 vs. 2 debate cycles
    \item Agent tools: individual tool contribution
\end{enumerate}

\section{Writing Timeline}

\subsection{For ACL 2025 February Deadline}

\begin{center}
\begin{tabular}{@{}lp{8cm}@{}}
\toprule
\textbf{Date} & \textbf{Milestone} \\
\midrule
Week 1-2 (Dec) & Complete missing experiments, finalize results \\
Week 3-4 (Dec) & Write Methods section (Sections 3-4) \\
Week 1 (Jan) & Write Experiments and Results (Sections 5-6) \\
Week 2 (Jan) & Write Introduction and Related Work (Sections 1-2) \\
Week 3 (Jan) & Write Analysis, Discussion, Conclusion (Sections 7-8) \\
Week 4 (Jan) & Create all figures and tables \\
Week 1 (Feb) & Internal review and revision \\
Week 2 (Feb) & External review, final polishing, submission \\
\bottomrule
\end{tabular}
\end{center}

\section{Supplementary Materials}

\subsection{Appendix Contents}

\begin{enumerate}
    \item Full prompt templates for all judge types
    \item Detailed hyperparameter settings
    \item Complete results tables with confidence intervals
    \item Additional qualitative examples
    \item Dataset annotation guidelines
    \item Grammar RAG database samples
\end{enumerate}

\subsection{Code and Data Release}

\begin{enumerate}
    \item GitHub repository with full implementation
    \item Gold evaluation datasets (anonymized if needed)
    \item Pre-trained RAG databases
    \item Reproduction scripts
    \item Demo notebook
\end{enumerate}

\section{Risk Assessment}

\subsection{Potential Reviewer Concerns}

\begin{enumerate}
    \item \textbf{``Just prompt engineering''}: Counter with novel ensemble algorithms and agentic framework
    \item \textbf{``Limited languages''}: Emphasize language-agnostic design, show transfer
    \item \textbf{``High cost''}: Highlight escalation efficiency, cost reduction
    \item \textbf{``No theoretical contribution''}: Focus on empirical novelty, practical impact
\end{enumerate}

\subsection{Mitigation Strategies}

\begin{enumerate}
    \item Strong ablation studies showing each component's contribution
    \item Comparison with diverse baselines
    \item Human evaluation correlation
    \item Production deployment case study (if available)
\end{enumerate}

\section{Conclusion}

This paper plan outlines a comprehensive strategy for publishing AQUA at ACL 2025 or equivalent top-tier venues. The key strengths are:

\begin{enumerate}
    \item \textbf{Strong novelty}: First agentic ensemble approach for GEC evaluation
    \item \textbf{Solid empirical results}: 93\% 4-class accuracy, significant improvements
    \item \textbf{Practical impact}: Production-ready, cost-efficient, multilingual
    \item \textbf{Comprehensive evaluation}: Multiple languages, ablations, analysis
\end{enumerate}

With execution of the experimental plan and adherence to the writing timeline, this work is well-positioned for acceptance at a top NLP venue.

\end{document}

